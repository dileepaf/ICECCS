\section{RELATED WORK}

Nash-equilibria is a popular topic in game theory. Algorithms exist for finding nash-equilibria in games~\cite{FPT04, DMP07, Daskalakis13}. However, these works usually use simulation and mathematical analysis techniques~\cite{BFW06}. Model checking has the advantages of both simulation, being automatic, and mathematical analysis, covering all possible behaviors.
Hence, Model Checking has been applied to verify game theoretic properties, for example~\cite{Henzinger98, AM04, LR06}. Among these works, we highlight the following which focused on probabilistic behaviors and automatic model checking tools for games.

The model checker MOCHA~\cite{AHM98} has been widely used for verifying games, e.g.,~\cite{KR03, MS05, ZZPM12}. However, nash-equilibria checking is not explicitly supported, and probabilistic behaviours are not supported.
Recently, a tool PRALINE has been developed specifically for computing nash-equilibria~\cite{Brenguier13}, and a tool EAGLE is developed for verifying nash-equilibria for concurrent games~\cite{TGW15}. However, these tools do not support probability.

Model checker PRISM~\cite{KNP11} supports probabilistic behaviours, but does not explicitly support verifying game properties.
Using the Rubinstein's protocol as a case study, Ballarini et al. show that probabilistic model checking (using the model checker PRISM) can be used for game analysis~\cite{BFW06}. This work inspirits automated probabilistic model checking for games, but it is specific to the negotiation game (a bargaining between a buyer and a seller) framework and does not address nash-equilibria analysis.
Later, PRISM-Games extends the model checker PRISM to be able to verify turn-based games~\cite{CFK13}. PRISM-game is able to verify nash-equilibriums, but only for turn-based games.

Probabilistic model checking has been used to analyse multi agent dynamics \cite{hao2012probabilistic} . Two strategies to arrive at maximal dispersion outcome (MDO) \cite{hao2012probabilistic} in a generalized anti-coordination game has been discussed. Counter Abstraction is used to reduce the state space. MDO can be seen as some form of equilibrium of the multi agent game, but it is not the formal Nash Equilibrium. Hence, this work is an application of probabilistic model checking to a specific class of games and it does not generally address the concept of nash-equilibrium.  


%PRISM-Games~\cite{CFK13} and negotiation-games~\cite{BFW06} describes turn based games in abstract level and %application level respectively. They use CTL to define parametric game properties which are not limited to Nash-equilibrium. %Both works use PRISM model checker for verification.

%Negotiation games presents a case study on “Rubinstein’s protocol” which is a protocol defined for bargaining between a buyer %and a seller. Two pure strategies for the game are, accepting the most recent offer made by the opponent player or coming up %with a new offer. This case study analyses how mixing these strategies according to some probability distribution, result in %different expected utilities. Here model checking is used to derive the reaching probability of different offers according to a %predefined acceptance probability function. Negotiation game is a special form of a game where the strategy space is %continuous %hence cannot be formulated as a normal form game and as a result, Nash equilibrium cannot be found with %classical game theory %algorithms. They claim, using probabilistic model checking for this analysis as a novelty and their goal %is to extend this work to %the level of finding Nash equilibrium.
%
%Tajouddine et al.~\cite{TGV08} Improves the state space explosion in SPIN model checker by using abstractions for the %verification of game theoretic properties of auction protocols. This verifies the strategy-proof ness of “Vikrey auction” with a %novel abstraction technique. This abstraction is unique in the sense that it is defined based on the property to be verified.

The most relevant work is the one by Mari et al.~\cite{MMS08}. This work proposes a symbolic model checking algorithm for verifying nash-equilibrium in distributed cooperative systems. It is the first algorithm to automatically verify that it is in the best interest of each rational agent to follow exactly a given system~\cite{MMS08}. A player's all possible behaviour is modelled as a finite state mechanism. The given system describes which action should be taken in which situation (by honest/altruistic players), which is modelled as constrains on state transitions of a game tree. Infinite sequences of actions are tackled by using \emph{discount factor}~\cite{FT91} to decrease relevance of rewards that are far in the future, which allows them to only look at finite sequences of actions. The authors propose a notion of equilibrium which takes both \emph{Byzantine} players -- players that randomly take possible actions and ignore utilities/rewards, and a \emph{tolerance} of rewards -- the reward difference between following the system and deviating from the system, as parameters; and prove that using their algorithm the proposed equilibrium can be automatically verified by just looking at finite sequences of actions. This algorithm does not support verification of probabilistic systems. We adopt a similar specification framework of players and equilibrium, and develop nash-equilibria verification algorithm, which focuses on probabilistic systems. Even though we define the verification algorithms for distributed systems with cooperative actions, the algorithm can be easily extended to support distributed systems with turn based actions as well.

% discusses the importance of using model checking for finite concurrent systems. Classical game tree mechanisms usually fail in %this scenario due to large (possibly infinite) state spaces. Since the state space is finite, model checking techniques become %more useful in verifying game theoretic properties in these systems. This work verifies nash equilibriums with arbitrary errors %in distributed system protocols in the presence of byzantine nodes using a symbolic model checker called NuSMV. However, this %work does not handle probabilistic protocols.

%Rational cryptography is an emerging area which analyses cryptographic protocols. Main aim of these is to design security %protocols such that the best choice of a self-interested agent is to follow the protocol instead of violating it. Automated proofs in %rational cryptography is not trivial because it has a security proof as well as a game theoretic proof. Backes et al.~\cite{BCM13} %defines a Nash equilibrium verification tool (ProVerif) for automated verification of rational cryptographic protocols. Applied pi %calculus is used to model these rational cryptographic protocols. This is the first effort of automated verification of rational %cryptographic protocols. However, their verification is specific to rational file sharing protocol.\nd{the paper seems not published}
%
%Defining high level languages to model, classes of game theory problems has also been investigated. ‘rPATL’ is such a language %defined in 2013~\cite{CFK13}. This language is specifically designed to handle turn based stochastic games.
