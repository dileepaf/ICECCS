\section{SPECIFICATION}
\label{sec:mech} 
%Our analysis consider the parallel synchronous systems. This means that all the nodes/players in the system perform their state %transitions at once. Distributed systems which perform protocol steps collectively (not strictly synchronous) can also be %
%included in the above category. In this setting we have two kinds of players namely, Byzantine and Altruistic.

%A parallel synchronous system can be seen as a system which goes through state transitions. Each player performs a transition %to a local state based on his global state and the available actions from that global state.

As mentioned in the introduction, we consider distributed systems with multiple players who need to cooperate in order to achieve a goal. Games for such systems are inherently distributed with multi-players, rather than turn based. In addition, a system can be non-terminating, i.e., there are infinite executing steps for some players. For instance, a system (e.g., the case study protocol) can be executed unbounded number of times.

Games are often represented in a normal form -- a matrix of players, strategies and payoffs, or in an extensive form -- a finite game tree specifying the sequences of actions. Such specification is no longer suitable when the game has infinite execution steps. In order to model infinite games, we adopt the finite state mechanism representation defined in~\cite{MMS08} -- each player is specified as a finite state mechanism, potentially with loops. This allows us to specify infinite execution in a finite manner.

%In a game theory setting, actions of each player depends on the amount of information they have. Practically, a player may not %have full information about other player states and their actions.  But he should have sufficient information about the global %state to perform the state transition. This modeling and analysis happens in a third party point of view who can view all the %global information. We can draw a local finite state machine for each player to list out the possible actions and its local states. %However, this can be an under specification because, in some scenarios local state cannot decide whether it can perform an %action specified in the local finite state machine.

%\subsection{SPECIFICATION}

We use the formalism of finite state mechanisms~\cite{MMS08} to model all possible local states and local actions of players. All players' local states and actions together form all possible global states and actions in a game. 

Our contribution in this section is, extension of the finite state mechanism in \cite{MMS08} for probabilistic settings. We describe, how we changed the non-probabilistic representation to arrive at a probabilistic representation. First of all, we update the model of players to be probabilistic. Then we extend the state representation to be probabilistic. More importantly, we present the calculation of the value $v_i^{k}(Z, s)$ and worst value $u_i^{k}(Z, s)$ of a state in the probabilistic settings.

We start from general properties of players and end with specific properties of player types. 
Each player $i$ has a set of local states, denoted by $S_i$, a set of initial states, denoted by $I_i$, and a set of actions, denoted $A_i$.  Action set representation would follow from the definitions in section \ref{subsec: game_theory}.

Set of all the local actions can be denoted by $\hat{A} = \cup A_i$. Before defining the transition function we must define different player behaviours. We also introduce synchronized actions. The set of synchronised actions can be denoted by $A^s$. $A^s \subseteq \hat{A}$ 



\paragraph{Byzantine Player}
In a non-probabilistic setting, a Byzantine player is modelled as all possible local states and their corresponding feasible local actions.
In a probabilistic setting, a Byzantine player has exactly the same local states and the corresponding local actions, since probabilistic systems do not introduce additional local actions.
That is to say, similar as in the non-probabilistic setting, a Byzantine player can deviate from a system specification by performing actions that are not defined in the system.
In addition to perform extra actions, for the probabilistic actions in a system, a Byzantine player is able to deviate from the specification by changing the probability distributions of the actions. For example, if a system specifies a player's action as follows: the player toss a coin - with probability $1/2$ to be head and $1/2$ to be tail, if it is head, the player sends $1$, otherwise, sends $0$. A Byzantine player can deviate from the specification by changing the coin into a special one -- with probability $1/3$ to be head and probability $2/3$ to be tail. Furthermore, a Byzantine player is able to assign probability to non-probabilistic actions. For example, a system specifies a deterministic action of sending $1$, a non-probabilistic Byzantine player can deviate by sending $0$, a probabilistic Byzantine player can toss a coin -- with probability $\alpha$ to be head and probability $1-\alpha$ to be tail, and sends $1$ if it is head and sends $0$ if it is tail. To summarize, a Byzantine player is able to assign an arbitrary probability distribution (which adds up to $1$) to at a local state.



\paragraph{Altruistic Player}
An altruistic player follows system specification, and thus has predefined probability distribution for each probabilistic action. A probabilistic altruistic player covers the definition of non-probabilistic altruistic player as well. Each state has a unique probabilistic action defined. If only one action is defined it can be considered as a specific case of containing a single action of probability $1$.
%Locally, this can be modelled by adding the pre-defined probability to probabilistic actions and keeping other actions the same as in the non-probabilistic settings.

%Globally, in order to specify the hybrid actions, we need to distinguish which actions are probabilistic and which are not. Since the system specifies the distinction. At a global state, we are able to tell the probabilistic actions from the non-probabilistic actions.

Since local actions at a local states are feasible actions at the corresponding global states,

%this can be formally modelled by defining a function $P_i: S \times A_i \rightarrow \mathbb{R}$ where $\mathbb{R}$ ranges over $[0,1]$, with the requirement that
%at a global state $s$, \(\Sigma_{a \in A_i \wedge B_i(s,a)} P_i(s,a) = 1.\)

We model probabilities of performing local actions starting at given local states. Define $P_i: S_i \times A_i \rightarrow [0,1]$ , with the requirement that
$\forall s_i \in S_i$, \(\Sigma_{a_i \in A_i} P_i(s_i,a_i) = 1.\)

A local action $a_i$ is defined in local state $s_i$ $ \iff P_i(s_i,a_i) > 0$.

\paragraph{Rational Player}
A rational player is a special case of a Byzantine player who has the full power to change the probability distribution of actions at a state, driven by utility. We show that a rational player would always choose an action which gives the maximum utility with probability $1$.
Taking the simplest case with two actions,
\begin{itemize}
	\item If both of the actions lead to paths that give exactly the same utility, the probability distribution of the two actions does not influence the utility. Hence, we can assume the player choose one action with probability $1$, the other with probability $0$, without loss of generality.
	\item If the two actions lead to paths that give different utility, the player would choose the one gives bigger utility with probability $1$, the other with probability $0$.
\end{itemize}
The same reasoning holds for the case that the player has more than two feasible actions.
Therefore, we can safely model a rational player with non-probabilistic actions.


\paragraph{Local State Transition}
Function $B_i: S\times A_i\times S_i \rightarrow \mathbb{B}$ is used to specify the feasibility of player $i$ performing a action $a\in A_i$, which leads $i$'s local state to $s'\in S_i$, at a global state $s\in S$  -- if it is feasible, $B_i(s,a,s')$ is true, otherwise false.
The function $B_i$ is required to have two properties: \emph{serial} and \emph{deterministic}:
\begin{itemize}
	\item
	Serial property says that there is at least one action defined for each player at each global state, i.e.
	($\forall s \in S$, $\exists a \in A_i , \exists s' \in S_i$ s.t. $B_i(s,a,s')$ is true);
	\item
	The deterministic property says that a player's action $a$ should lead to a unique resulting local state, i.e. ($(B_i(s,a,s_1)=true \wedge B_i(s,a,s_2 )=true) \implies s_1 =s_2$.)
\end{itemize}
Due to the deterministic property,  we simply write $B_i(s,a)$ to represent that action $a$ is valid in a global state $s$, since the local state $s'$ is uniquely defined. Similarly, the global possible actions of all players are $B = B_1\times B_2\times\cdots\times B_n$.

%Using the above definitions, a  player's is defined as as a tuple $\langle S_i, I_i, A_i, B_i\rangle$. All players together actually form a game which is formally defined as the tuple $\langle S, I, A, B\rangle$. Note that adding probabilistic behavior does not change the actions and the states. Hence, we adopt this formalism and extend each action with a probability, see Section~\ref{ssec:prob}.

In non-probabilistic settings as well as probabilistic settings,
$\langle S_i, I_i, A_i, B_i\rangle$ defines a Byzantine player, since a Byzantine player can perform all actions that are feasible at a state, ignoring utility/rewards.
%Byzantine players can be used to model the ``bad guys", for example, the Dolev-Yao attackers for security protocols.
%
Recall that a system specifies which action should be performed at which state by which player, i.e., a system is a specification of altruistic players.
Similar to $B_i$, function $T_i : S \times A_i \rightarrow \mathbb{B}$ is defined to specify a player $i$'s behaviour of genuinely following the system -- $T_i (s, a)=true$ ($s\in S$, $a\in A_i$) if $i$ shall perform $a$ at state $s$ according to the system.
Since following the system (altruistic player) is always part of the possible behaviour, we have $T_i (s,a) \implies B_i (s,a)$. In addition, we assume the system does not have unspecified situation or deadlocks. Formally, we require that $T_i$ is non-blocking (i.e. $\forall s \in S,  \exists a \in A_i$, s.t. $T_i(s,a)$).
A system specifies the altruistic players, formally specified as $T=\langle T_1, \cdots, T_n\rangle$.%Note that there can be multiple valid actions at a global state.

Since that function $T_i$ models the altruistic player's behaviour, and $B_i$ models a Byzantine player's behaviour, given a set of  Byzantine players (denoted as $Z$) among the total $n$ players, the behaviour of Byzantine and altruistic players is summarised as:

\begin{equation*}
BT_i(Z,s,a_i,s'_i)=
\begin{cases}
B_i(s,a_i,s'_i) &  i \in Z \\
B_i(s,a_i,s'_i) \wedge T_i(s,a_i) & i \notin Z
\end{cases}	
\end{equation*}

\paragraph{Payoff Function}\cite{MMS08}

Player $i$'s payoff is defined as a function $h^{\psi}_i: S_i \times A_i \rightarrow \mathbb{R}$, i.e., for each global state transition and a global action, there is a payoff value for player $i$, denoted as numbers in $\mathbb{R}$. Here, ${\psi}$ is a set of global constraints. 
%The issue is that, each global state and global action pair
%may not have a payoff defined because some actions can be attributed to states which they have no connection with.
%   \cite{MMS08}Even though hi is defined for all the (global state, global action) pairs, in our setting, only useful information %would be given by the hi values defined for valid state transitions. Practically it may also not be meaningful to define pay-off %values for invalid state transitions.
Hence, $h = \langle h_1,\cdots,h_n\rangle$ denotes the payoff of all the $n$ players at any state for performing any action.
The payoff function can be directly adopted to the probabilistic settings since it is not influenced by adding probability.

\paragraph{Discount factor}
Let $\beta_i \in (0,1)$ is the discount factor for player $i$. The discount factor is used to decrease the relevance of rewards decreases in the future, so that to ensure efficient computation of a finite strategy in the settings of infinite games. We adopt the discount factor as well in order to handle infinite games in probabilistic settings.

\paragraph{Local Finite State Machine}
The local finite state machine for player $i$, in a probabilistic setting can be defined as $(S_i, I_i, A_i, BT_i, h_i, P_i )$

We can see how the local state machines can be combined to come up with the global state machine. 
$(S, I, A, h) =(\Pi_i S_i, \Pi_i I_i, \Pi_i A_i, \Pi_i h_i)$. 

The remaining components are $P_i$ and $BT_i$. \newline

We can define probabilities of global actions given a global state.
Global state $s \in S$ can be written as a local state sequence, $\langle s_1,s_2, \dots s_n \rangle$ and global action $a \in A$ can be written as local action sequence $\langle a_1,a_2, \dots a_n  \rangle$ as defined above. 
Now we can define, Probability function for global actions $P : S \times A \times \mathbb{P}([n]) \rightarrow [0,1]$. \newline

\begin{equation*}
P(s,a,Z)=
\prod_{i \notin Z} P_i(s_i,a_i)
\end{equation*}  

The global transition function $BT$ also should be updated to support synchronized actions.
We define set $J_i$ ,for $i \in 1 \dots n $  to count the number of occurrences of synchronized local action $a_i$ in the action sequence $a$.
\newline
$J_i$ = $\{k$  $ \vline$ $ k \in 1 \dots n \wedge a_i \in A^s \wedge a_i = a_k \}$. \newline

\begin{equation*}
BT(Z,s,a,s')=
\begin{cases}
\wedge _ i BT_i(Z,s,a_i,s'_i) &  \forall \text{ } i \text{, } a_i \notin A^s \\
\wedge _ i BT_i(Z,s,a_i,s'_i) \wedge (J_i(a) != 1 )  & otherwise
\end{cases}	
\end{equation*}


It should be noted that if $BT(Z,s,a,s')=false$ then  $s'$ is a deadlock state. Now, summarized global finite state graph in the probabilistic setting is defined by,
$(S, I, A, BT, h, P )$.

We can move on to value calculations on the global finite state graph. We would need two concepts called path and strategy.

\paragraph{Path}
With the above definitions, a non-probabilistic game can be specified as $(S,I,A,B,T,h,\beta)$. Given such a game with a set of Byzantine $Z$,
a path is defined as a sequence of possible $(s,a)$ pairs where $s \in S$ and $a \in A$. We denote a path as $X=\langle X_0, \cdots, X_n, \cdots\rangle$, where each step $X_t$ ($t\geq 0$) is a $(s,a)$ pair. We use $X_t^{(s)}$ to denote the first element of a step in a path -- the state $s$ and $X_t^{(a)}$ to denote the second element -- the action $a$. Path length is the number of global actions contained by the path.
A path $X$ is valid if $\forall t < |X| $, $BT(Z,X_t^{(s)},X_t^{(a)},X_{t+1}^{(s)})$ is true, meaning that the the transition from one step to the next step is valid.
% where $s_i(t)$ and $a_i(t)$ are the local state and local action at step $t$ in a path respectively.
%Extra notation can be introduced to extract local states and actions from global states and actions.

In non-probabilistic setting, the utility of a player $i$ in a path is evaluated by adding up the discounted payoff values along the path, formally,
$V_i(X)= \Sigma_{t=0}^{|X|-1}{\beta_i^{t} h_i(X_t^{(s)},X_t^{(a)})}$, where $|X|$ is the path length, which can be infinite. When the path length is infinite, $V_i(X)$ converges to an upper bound, because of $\beta_i$.

$Path_k(s,Z)= \{\pi | \pi \text{ is a path in }(M,Z) \wedge |\pi|=k\text{ and }\pi^{(s)}(0)=s \}$
%$Path_k(s,Z)= \{X | X$ is a path in $(\mathcal{M},\mathcal{Z})$ and $|X| = k$ and $X_0^{s} = s\} $
%This set defines the paths of length $k$ which starts from state $s$.
%These paths can be filtered by the maximum cardinality of the set of Byzantine agent that is allowed.
%$Path_k(s,f) = \cup_{|Z|\leq f}$ $Path_{k}(s,Z)$. When the path has infinite length, we can omit the subscript $k$.
In a probabilistic setting, a path denotes a set of paths that are reachable through unique probabilistic strategies of Altruistic players and a strategy of Byzantine players. A path tree can be denoted by $\{\pi\}$. A path tree of length $k$ is denoted by $\{\pi|_k\}$
The concept of a probabilistic path can be defined through the concept of probabilistic action. Each non deterministic action choice by the set of Byzantine players correspond to one path tree in the global graph.

\paragraph{Strategy}

To reason on a game, it is often needed to distinguish one player's actions from other players. Hence, the notion of strategy is defined as follows: A strategy of a player is a sequence of local actions of the player. A strategy at a state $s$ defines what a player would behave in the future. Depending on other players' actions, there will be multiple paths with multiple utility values. The minimum utility value is called the \emph{guaranteed outcome} of the strategy for the player at the state $s$. In the algorithm in~\cite{MMS08}, two important values are calculated: 
\begin{itemize}
	\item the \emph{value} of a state $s$ at horizon $k$ for player $i$ with Byzantine players $Z$, denoted as $v_i^{k}(Z, s)$, is defined as the guaranteed outcome of the best strategy of length $k$ starting ate state $s$;
	\item the \emph{worst case value} of a state $s$ ate horizon $k$ for player $i$ with Byzantine players $Z$, denoted as $u_i^{k}(Z, s)$, is defined as the guaranteed outcome of the worst strategy of length $k$ starting at state $s$.
\end{itemize}
In a probabilistic setting, a strategy is defined the same as in the previous section. The only difference is that all the strategies of an Altruistic player are assigned a probability. Strategies also appear as sets. $\forall path set \{\pi \}$
there is a corresponding strategy set $\{\sigma \}$ for player $i$. 

In addition to the state machine representation, we would add two definitions for global probabilistic action and a global probabilistic state.

\paragraph{States}
We extend the global state representation in~\cite{MMS08} with probability. We can consider resulting states of a probabilistic action as a single probabilistic state.

\paragraph{Actions}
Since some players are probabilistic, a global action is defined as a set of atomic actions where each action can be assigned a probability. 

Let $a \in A,$ $a_Z$ be the local action components of Byzantine players and $a_{[n]-Z}$ be the local action components of altruistic players. A probabilistic global action can be defined as, \newline

$ GA=\{ a \in A | \forall a',a'' \in GA, a'_Z = a''_Z  \wedge \cup_{b \in GA} b_{[n]-Z}=A_{[n]-Z} \wedge  \forall c',c'' \in GA, c'_{[n]-Z} = c''_{[n]-z} \implies c'=c'' \}$


An example action is presented in figure \ref{fig: prob_act}. Let $s= \langle s_1,s_2,s_3 \rangle \in S$. Let there are $3$ players $1,2,3$ with enabled action sets $\{a_1,a_2\}$, $\{b_1,b_2\}$ and $P\{c_1,c_2\}$ respectively. Let $1$ be an altruistic player. \\
$P_1(s_1,a_1)=p_1; P_1(s_1,a_2)=p_2;$

\begin{figure}
	\begin{tikzpicture}[->,>=stealth',shorten >=1pt,auto,node distance=2.8cm,
	semithick]
	\tikzstyle{every state}=[fill=red,draw=none,text=white]
	\node[initial,state] (A)                    {$s$};
	\node[state]         (B) [above right of=A] {};
	\node[state]         (C) [below right of=A] {};
	\node[state]         (D) [right of=B] {};
	\node[state]         (E) [right of=C] {};
	\node[state]         (F) [right of=D] {$s'$};
	\node[state]         (G) [right of=E] {$s''$};
	
	\path (A) edge              node {$a_1$,$p_1$} (B)
	edge              node {$a_2$,$p_2$} (C);
	\path (B) edge              node {$b_1$} (D);
	\path (C) edge              node {$b_1$} (E);
	\path (D) edge              node {$c_1$} (F);
	\path (E) edge              node {$c_1$} (G);
	\end{tikzpicture}
	\caption{Probabilistic Global Action Example}
	\label{fig: prob_act}
\end{figure}
This probabilistic action can be written in set notation, $\{(a_1,b_1,c_1),(a_2,b_1,c_1)\}$. The other possible probabilistic actions are, \\
$\{(a_1,b_1,c_2),(a_2,b_1,c_2)\}$, $\{(a_1,b_2,c_1),(a_2,b_2,c_1)\}$ and $\{(a_1,b_2,c_2),(a_2,b_2,c_2)\}$.
same strategy of rational players is applied to both probabilistic choices because of the parallel synchronous nature of the system. Rational players do not have information about the probabilistic choice made by the altruistic player.