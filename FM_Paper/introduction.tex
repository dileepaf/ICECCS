\section{INTRODUCTION}%need to be capitalized by UAI
% cooperative, concurrent, multi-player, probabilistic, finite state mechanism, perfect-inf,  

Many systems in the real-world involve collaboration of many distributed parties, e.g., Internet routing~\cite{SP04}, peer-to-peer file sharing~\cite{BCK10}, cooperative backup~\cite{LEB03}, etc. 
In many of such systems, participants may have conflict interests. Thus, it is natural to consider each party as rational, meaning that each party only collaborates if his benefit from collaborating is more than not collaborating. In this sense, the system is actually a game. 
An important property for such systems/games is to ensure that every rational player would not deviate from the expected behavior specified in the systems. That is to say, to pursue the most benefit, the rational players would all follow the system, i.e., the system is the nash-equilibrium in the concept of game theory -- no one derivate from the system because this would harm their benefits.

Traditionally, when designing a system, participants are assumed to be either ``good guys'' -- who follow  the system behavior, or ``bad guys'', who do everything in their power to break the system~\cite{AAH11}. Rational/selfish participants are seldom considered, and few systems\footnote{Notable exceptions include~\cite{AAH11,LCW06}} work correctly under the rational settings (as stated in~\cite{MMS08}). 

When the desired properties, such as security or reliability, of such systems are critical, whether every collaborator following the system becomes of great importance. In this case, precisely proving that the system is indeed the desired equilibrium is necessary. Such proofs are often manual, e.g., proofs in the contracting signing protocol~\cite{Rabin83} and the secret sharing protocol~\cite{HT04}. As the proof tasks are often heavy, manual proof is error-prone and inefficient, especially the designer of a system may not be an expert on formal proofs. Thus, automated proof is desirable. Model Checking\cite{BK08}, which is a precise way to verify systems and has many automated tool support, is a promising method.

Automated model checking tools for games exist, e.g., MOCHA~\cite{AHM98}, MCMAS~\cite{LQR09}, GAVS+~\cite{CKL11} and PRISM-games~\cite{CFK13}. The existing tools verify various types of properties in various types of games. However, none of them is able to verify the nash-equilibrium type of property -- whether following the system is the optimal strategy for each rational players, in concurrent games. Recently, an algorithm has been developed to verify such properties in a sub-class games which can be described using finite state mechanisms~\cite{MMS08}. However, this work only considered games without probability. In reality, many systems have probabilistic behaviours.
Some of the systems are inherently probabilistic. Economic and biology systems are examples. There are other systems that fail to achieve certain goals without involving probability, for example, it is proved that non-probabilistic secret sharing scheme is not practical in the rational settings~\cite{HT04}. Hence, we propose an approach to verify nash-equilibrium for probabilistic systems. 

\paragraph{Contributions.}
We develop a verification algorithm to verify whether a probabilistic game is a nash-equilibrium. 
We implement the algorithm based on PAT (Process Analysis Toolkit), which is a model checking platform providing efficient general probabilistic model checking algorithms~\cite{PAT}.
We perform a case study on analyzing a probabilistic secret sharing scheme~\cite{HT04}. The authors of the scheme first prove that no practical mechanism for secret sharing works without using probability in the rational setting and further provide the first secret sharing with the rational participants in mind. 
This case study first illustrates how our algorithm and tool help to ease the proofs of the non-existence of non-probabilistic scheme. Second, it shows how the tool automatically verifies the correctness of the probabilistic scheme. In addition, during verification, weaknesses of the probabilistic scheme have been found.